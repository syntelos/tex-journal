\input preamble

The social dynamic of fear and its reflections and delusions produces
cyclic psycho-behavioral force field structures.

Fear in the social domain associates strongly with sources of
frustration.  The fear of fear is the individual social complex.  The
adaptation to such conditions includes identity and grouping behavior
that spans spectra from pathological to compassionate.  

With an appearance of a reactant, the psycho-intellectual solution is
a reaction that is capable of attacking anything that is capable of
absorbing, socially, the behavioral appearance of the fear complex.
The sources of the fear complex are submerged below the characteristic
reaction behavior and its mitigating adaptation reaction behavior.
The identity-group normalization or hysteric behavior has the group
dynamic effect of accepting the externalization of the appearance of a
problem.

These phenomena of social fear complexes are amplified by
communication technology applied to social dimensions.  The individual
and group may break the behavioral cycles that reinforce these
psychological complexes with their respective self-awareness.  

They would succeed in doing so by recognizing the extent of the
refactoring of their situations to these dimensional horizons.  The
individual horizon may never have been recognized as local, not
global.  That is, a vague knowledge of global social issues has
impacted the local individual perspective to such a degree that
identity and group dynamics have been employed in adaptation --
perhaps over the entire life experience.

The effect is a catastrophic personal impact.  In many cases the
individual is locked into an identity - group complex of such ferocity
as to obscure reason.  The identification function colors persons and
groups as exteral or other, obscuring society.

\bye
