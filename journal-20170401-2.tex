\input preamble-png

The qualification of perception as opaque or transparent has two
hemispheres.  In the opaque hemisphere is the perception of the
metaphysical world, and in the relatively transparent category the
perception of the physical world.  In the metaphysical world are the
characters and derivations of others, {\it animus}.  In the physical
world is nature, {\it inanimus}.

There exists a kind of combat within the metaphysical world of people
and their actions.  There are those who are willing to explain or
represent themselves, and those who are not, those whose
representations are always opaque and fleeting.  In the former I have
only ever found discussion and openness, and in the latter I have only
ever found denial and avoidance.

Of course it is impersonal to look at ``the world'' this way.  It has
a purpose which is lost when removed from its analytical frame.  That
purpose is to share the self possession of an opacity of view which is
variously tinted, and to declare it not only normal but necessary.  It
is a psycho-perceptual necessity common to everyone, that metaphysical
perception is opaque, not transparent.

In art and science we endeavor to remove elements of the metaphysical
realm of animate opacity into the physical realm of inanimate
transparency.

\bye
