\input preamble

The force within has an unstoppable character.  Let's call it the life
force.  Obviously my force is most commonly known by others as the
force of my body, which has very well known strengths and weaknesses.
The study of the strengths and weaknesses of the body has been a
central concern across human endeavor with and without the human
discipline and regard we typically call science: the impartial context
that separates subjective from objective.

The life force transcends the body by media including communication
and memory.  In the metaphysical conflict, the object of {\it other}
-- the body and its force -- is the subject of jealosy and envy.  In
the metaphysical peace, the object of {\it another} is the subject of
sharing and contribution.  It is clear enough from the experience of
an adult in the millionth generation of the {\it homo sapien} that
conflict is wasteful and peace is productive.  It is also known to be
true by the definition of terms and a context free extrapolation of
symbols.  

In other words, that's plainly obvious.  We nurture a state of
vigilance over interpersonal conflict, and fail to develop the skills
necessary to winning.  That's both counter productive and self
contradictory.  

These primitives are filling voids left behind by the loss of the arts
of interaction and communication.  This partial expression dominates
substance and significance in the recognition of the history of
situation and adaptation.  Former arts of interaction and
communication have been displaced by the technological evolution and
its counter-intuitive effect on education and development.  That is,
the dramatic rise in human technology of living has produced a
comparative loss of human art of living.  As a result a gap or void
exists that is filled by the primitives inherent in the fundamental
choice of peace versus conflict.  In other words, ignorance and
confusion reins where facility is lost.  We know this.  We don't like
it, but we know it to be true.  We know the discomfort of profound
uncertainty.  The sense that the more we know, the less we know.  This
is no less than the sense of the lack of scholastic discipline.  With
the discipline of mind and body proportional to the technologies of
life we find comfort -- we find an occupation and we find comfort --
and this extrapolates from the ground of birth into the cosmos of the
not yet known.


\bye
