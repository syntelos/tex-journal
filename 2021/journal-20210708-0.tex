\input shorts

{\title Logical existentialism}

{\date Thursday, 8 July 2021}

{\section Cognitopathology}

\bigskip

With the definition of the sociopathology as the study of
objectification, we would want to define a cognitopathology as the
study of rationalization.

The principal feature of the cognitopathology would be the processes
of actualization and realization.  A cognitive strategy might not
afford practical actualization.  Features of actuality may be praised,
accepted, ignored, rejected, or condemned.  We like to believe that
everyone knows who the President of the United States is, and we know
that it is not universally true that everyone knows who the President
of the United States is.

A cognitive strategy of awareness implies practical processing of
actuality as intake and reality as production.  One might not
prioritize the cognitive I/O processes as important.  One might not
conceive of cognitive I/O processes.  One might conceive of body as
entirely automated, or one might not conceive of body as
metaphysically independent.  

\medskip

{\section References}

\medskip

{\ninerm [2021/02/15 Journal 0] A logical psychological morality}

{\ninerm [2021/06/02 Journal 0] Physical cognitivism}

{\ninerm [2021/07/06 Journal 0] Sociopathology}

\bigskip
{\tail John Pritchard, @syntelos [CC-BY-NC]}

\bye
