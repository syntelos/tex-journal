\input shorts

{\title A logical psychology}

{\date Tuesday, 9 February 2021}

{\section Kindergarten}

\body

A child's first step into the world of society occurs in kindergarten,
with the recognition of other as analogue to self.  As, ``the others
are the same as you'', or, ``he is just like you''.  When the
reflection of other as person becomes distinct, self as person becomes
distinct.  Presence becomes a reflection upon behavior.  And the vista
of the horizon of awareness opens.

It is necessary for the subjective being to open to the objective by
recognition.  The others are not objects subject to mind.

Otherwise, this ``graduation from kindergarten'' might never occur.
There are many adults who daily make the errors of awareness.

Ploys of object and association are performed by ``the invisible
man'', with an inpunity of ignorance that evaporates at the sight or
sound of a recognition of substance and significance.

\bye
