\input shorts

{\title Socialism as social independence}

{\date Wednesday, 10 February 2021}

{\section Of bonds and boundaries}

\body

It often occurs that one or both members of a couple will experiment
with the boundaries of fidelity.  It is another confirmation of the
existence of the romantic bond.  And the experiment is a test of the
existence and character of the romantic bond.  The opportunity to
exercise the romantic senses depends on practical trust as boundaries.

It's a virtual certainty that a third, uncommitted partner is and will
remain virtually externalized.  The practice demands boundary
awareness.  Fidelity to acceptable boundaries of recognition and
interaction should ensure reasonable behavior of communication.  In
reasonable and acceptable commmunication there are no violations of
fidelity that are not practical to the experience of fidelity.

The experimental exercise may have a profound character which is quite
distinct from the need to experience love's boundaries.  When the bond
captures a flawed couple, one partner may need to develop a practical
awareness of that problem.  Typically a developmental constraint or
obstruction.  Bonds formed before developmental awareness has any
practical comprehension become harmful, vaguely analogous to an
exploitable power structure but distinct as biological rather than
economics or social.  In these cases the bond must be broken, but the
awareness of causation must first be established and that can require
extensive exercise.  The entire process is many years.  At least two.
And as many as twice the duration of the failing relationship.

That kind of slow motion failure is the most subtle and therefore the
most difficult.  One continues to develop as normal, and one has
failed to maintain the interest and concern of the effort of awareness
as a perpetual process of actualization and realization.  That growing
distance between subjective and objective can be illusory.  It is a
relational allusion, a subjective novelty, and can be a toy.  The
subjective bed includes allusions to compassion, affection, and joy.
It's identification and characterization tedious.  Which is causal to
the demand for ternary romantic intimacy which is possessed of a
formality of communication and boundary.

The need for a third partner is partial and spirituo-intellectual.  It
is not physical or sexual --- at least not in concern.  That romantic
interdependence is a kind of social interdependence.  Many kinds of
relationships have network structures to reinforce or to amplify
relationships explicitly or implicitly.

{\tail John Pritchard, @syntelos [CC-BY-NC]}

\bye
