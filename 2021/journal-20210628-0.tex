\input shorts

{\title Political criminology}

{\date Monday, 28 June 2021}

{\section LOVEINT + SNOWDEN}

\bigskip

If we accept Edward Snowden's assertion that his disclosure has
proportion to an {\it attack on democracy} as {\it popular
sovereignty}, then we must review his assertion that the shape of that
is a systematic corruption of law enforcement inclusive of the
perversion of signals intelligence and cyber operations.

This is widely recognized as a factual representation of actuality.
Terms and forms vary by context, perspective, and purpose.  The
assertion and its denial, disassembly, and distraction have become
entrenched.

Certainly that is a veracious description of actuality.  Accepting
that ``Snowden assertion'' as factual, we have the situation of the
moment to inspect and examine.  It is dependent.  If the Snowden
assertion is counterfactual or afactual, then it is vacant.

Which is that the Obama, Biden, Harris world is genuinely anticorrupt,
truthful, and varacious.  And that the corrupt world has the shape
asserted by Edward Snowden.

Having conducted the hypothesis from 2013 to 2021, I remain committed
to holding that to be the most reasonable appreciation of actuality.

\bigskip
{\tail John Pritchard, @syntelos [CC-BY-NC]}

\bye
