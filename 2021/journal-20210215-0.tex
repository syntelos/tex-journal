\input shorts

{\title Logical existentialism}

{\date Monday, 15 February 2021}

{\section A logical psychological morality}

\body

The {\sl logical psychology of development} proposes a uniform theory
of development as the walk into, or avoidance of, awareness.  The
horizons of awareness are physical and mortal, and metaphysical and
social.  With the acquisition of language the metaphysical horizon
opens.  The process of objectification identifies, and the process of
rationalization characterizes.

With the development of language a discipline of objectification would
avoid the distraction of entertainment and enable the process of
reflection and association.  Likewise, the development of a discipline
of rationalization would avoid distraction and enable distinction in
the work of reason.  The identification and characterization of the
experience of actuality and the terms of reality is the work of
reason.  The quality of veracity in the processes of actualization and
realization is {\bf morality}.

Therefore, a {\sl moral fabric of being} is found in the practice of
morality.  A perpetual balance of reflection, action, and interaction
affords a practice of awareness.  The moral fabric produced by the
processes of actualization and realization is the metaphysical
manifold that supports and informs the spirituo- intellectual
consciousness that practices awareness.  In this conception, it is
memory as the products of existence and being.  And our current
practical conception of memory is an abstraction.  Alternatively, a
micro- metaphysical musculature.

The nascent subjective bed is a subjective sleep which one exits by
waking into awareness by the exercise of actualization and
realization.  Each living, breathing, conscious individual is a
continuous process of actualization and realization.  Individual
processes of cognitive effort develop the cognitive processes of
objectification and rationalization to a moral fabric, or abandon the
cognitive processes to depravity.  This morality is a metaphysical
epistemological ontology of wellbeing, far from the historical
subjective phenomenological errors of awareness by abstraction.  The
intersections are plain, but are exclusive and not inclusive of our
profoundly flawed moral, theological, and psychological history.  A
measure of significance is readily available to conceptions of a
humanity of substance distinct from an inhumanity of insecurity, and
the corresponding rights of the individual as a socially independent
human being.  The sovereignty of humanity.  

A psychological pathology $(\psi)$ of sanity, inanity, and insanity is
well evident.  It is here represented with the additional degree of
freedom with two pair of cognitive processes: the physio- linguistic
objectification and rationalization $(\iota)$; and the cognitomoral
actualization and realization $(\alpha)$.  The condition of effective
development is thereby represented as existential, subsistential, or
nihsistential.  The existential is secure in the practice of
awareness, the subsistential is relatively uncertain of the
significance of reason to cognition, and the nihsistential is
uncertain of the significance of metaphysics to cognition.

The psychological existence finds
$$
 ( \psi = {\alpha}/{\iota} ) > 1,
$$
and the psychological subsistence finds
$$
 ( \psi = {\alpha}/{\iota} ) < 1 .
$$
Likewise, the psychological nihsistence finds
$$
 ( \psi = {\alpha}/{\iota} ) \ll 1 .
$$
The observation constitutes a measure of the maturity of objective
awareness as distance from the nascent subjective bed.  

The quality of veracity in the processes of actualization and
realization is morality.  This metaphysical morality is roughly
equivalent to the measure of metaphysical maturity.

The psychological pathology is not linear.  A metaphysical
epistemology must capture the social effect of nonlinear moral
development.  There are nonlinear developmental effects evident from
perpetual exposure to the nihsistential mystery.  Hyper-
objectification induces hyper- objectification where experience and
training raises development into a subsistential maturity.  The common
case is hyper- sexuality.  An alternative case is racism (hyper-
corporality).  The society necessary to yield nihsistential outcomes
from a public education situation includes exposure to familiarity
with absurdity and inanity.  A walk into an avoidance of awareness
produces general inawareness or anti- awareness typified by cognito-
linguistic hyper- objectification and hyper- rationalization.

A moral pathology is socially relevant.  We might recognize the {\it
backstabber morality} of hyper- objectification in a hyper- sexualized
context, and the {\it nigger morality} of hyper- objectification in a
hyper- corporalized context.  The significance is to induce
subsistential developmental outcomes, or nihsistential developmental
outcomes, respectively.  The hyper- corporal context yields
nihsistential outcomes where a subsistential outcome might be
expected.

Individual wellbeing suffers with some proportion to the experience of
hyper- objectification.  A limited experience may be disturbing, while
a saturated experience yields metaphysical illbeing.  Simple
metaphysical illbeing is depression, while the deterioration of
metaphysical maturity is plainly possible.

The lifelong, perpetual processes of actualization and realization
support the continuous, lifelong cognito- linguistic processes of
objectification and rationalization.  The daily adoption of experience
requires a metaphysical work effort which is qualified by veracity and
produces acuity.  Likewise, a daily practice of awareness requires a
balance of reflection, action, and interaction in the production of
our social reality.

We might identify a pair of moral classes that are constructive in the
development of metaphysical maturity.  The two might be sequential.
The first, {\it angel morality}\/, could be conceived as the
projection of self preservation onto conservation of person.  Angel
morality arrives at a pacific awareness of humanity.  From such a
practice, social morality would arrive at an expansive employment of
consciousness and intellect in a perpetual advance of metaphysical
maturity.  The sequence reflects the development of consciousness from
elementary to social in the first two decades of life.  This
realization would produce a generosity of spirit and intellect.  


{\tail John Pritchard, @syntelos [CC-BY-NC]}

\bye
