\input preamble

It's hard to speak subjectively because the subjective semantic
context is more complex, more particular than formal subjective or
objective contexts.  We typically negotiate a reference frame, and
proceed or not with a thought depending on the patience and experience
of interested parties.

We come into the subjective space between us dressed as we are,
carrying what we bring.  On increasingly larger scales of presence, we
arrive to the subjective space between us with the habits that may be
derived from one's experience consuming marketing, or reading books.
Broad spectra of perspectives and attitudes.  Projection words: want,
criticism, love, hate, and envy. 

The habit of the media space between us often lacks reflection.
Reflection words: should, could, would.  Media professionals may
attempt to inject and invite reflection, and conversation often
originates in and revolves around reflection, but the space between us
often lacks the material of reflection.  Media to grow and learn with.

The subtended patience of the experience of media or conversation
ranges from calm and thoughtful and reflective, to nervous and
energetic, to hypnogogic and agitated.  A first exercise in the art of
life could be the evaluation of choices and behaviors in this most
casual of corners of living.

\bye
