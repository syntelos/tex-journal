\input preamble

The epistemological horizon on our existential sphere meets cosmos and
consciousness in the want - need complex that drives the homo sapien.
The person needs food, and we often equate food literal with food
figurative (as metaphor for knowledge and art).  

The person depends on food, and feeds the risk of supply or magnitude
of demand acutely in the body - psyche complex that we call home.
Hence the want - need - fear complex associates with the food -
knowledge - art complex to form a person.  This person is well capable
of feeling fear, which enhances self caring and concern and may impact
performance.  This performance ranges from sight to recognition and
production.  With maturity the emotions are identified as independent
of psyche.  The intellect is capable of identifying and characterizing
features of self on any abstraction that doesn't intersect with
itself.  These exercises develop the balance within necessary to
satisfactory performance and production.

The psyche is capable of profound fear that has been described in one
form as a panic attack.  This profound fear arrests performance and
production when it manifests from experience without the psycho -
intellectual facility to balance or counter or squelch it.  

It is driven by the internal, experiential reflection sense that
operates over memory and psyche and body like a muscle.  It has give
and take like the web of musculature that controls a limb -- a leg or
an arm.  It cannot be annihilated, but it can be moved like any of the
limbs.

The social dimensions of individual experience must never be received
by the person as inhibitors to the use of one's own body - psyche -
intellect.  One must adopt the habit and experience such that life is
harmonized within.  

The first example of this would be found in the learning relationships
of the parent and child or teacher and student.  In learning
situations there are experiences ranging over knowledge and senses,
stimuli that may be categorized as positive and negative, or perhaps
measured.  Experience provides learning and is a blessing, in all its
forms.  

This positive acceptance underlines the statement that social
experience not be received as inhibitive.  It it may be novel or
reinforcing.

\bye
