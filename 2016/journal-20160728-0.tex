\input preamble

There is a class of homo sapien that fears consciousness, fears
itself.  It fears the irrational wanderings of the mind as the
primitive, without knowing its own significances.  That the irrational
have no significance, and the significances of the rational have a
large range.

This consciousness is aware of its ways and means.  It is aware of the
poverty of consciousness in contrast to the lifelong effort spent on
maintaining and building personal health as wealth.  And it is aware
of the poverty of consciousness in the evidence of violence.  

Intellectual violence is the evidence of the ignorance of
consciousness, the poverty of consciousness, and the failure to invest
within.  The cause of this failure is not something one knows by
comparison to oneself.  It may be known to the bearer as a problem of
external and internal means to raise oneself up from the ground of
childhood into the spaces of adulthood, into the potentials of mind.
One may compare other individuals and witness relatively apparent
examples of external and internal, environmental and physiological
failures to consciousness.  One may recognize the physiological issues
of mind wherein the significance of the irrational plays a role in the
health of a mind, and obviates the relatively abstract issue of
consciousness.

And one should always endeavor to recognize the environmental issues
of means.  In an environment in which the intellectual violence of
ignorance is commonplace, a young mind will find the achievement of
this wealth to be challenging -- the level of difficulty proportional
to intelligence and opportunity and the awareness of the potential for
inner wealth.  

Individuals persue consciousness as a parent protects its own child.
We gravitate to it naturally.  We can miss it when we value the
external to the exclusion of the internal.

\bye
