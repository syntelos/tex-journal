\input preamble

To describe the interior within, Jesus' {\it soul}, Jung's {\it vita
peracta}, we employ {\it immortal soul}, {\it inner consciousness},
and {\it subconscious}.  Through education including mediatation and
{\it Hara} we learn that the center of the self resides within.  

Many outward expressions of the body are the reflections of the outer
consciousness by the inner consciousness.  To know oneself one must
balance the inner and outer consciousness into a {\it sensible
consciousness}.

The interior within has an immortal quality.  It reads and speaks and
feels the outer consciousness exclusively.  The {\it intellectual}
facilities of the outer consciousness face the {\it emotional}
facilities of the inner consciousness, and {\it vice versa}.  In
another conception of the overall psychic structure, the outer
consciousness is one's (world) interaction interface and the inner
consciousness one's safety.

From within one knows the sense of one's world view in terms that are
profound emotional reflections of the outer consciousness by the inner
consciousness.  That is, {\it self esprit}.  

In the balance of the whole there are not two parts, but one.

\bye
