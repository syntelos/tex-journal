\input preamble

The character we play with is deconstructed into Fear.  Fear ruins the
presence of mind, and invests in the ego of self preservation.  In the
classical, constructed role, Fear lies and tortures in order to
disorient and exploit and thereby destroy the person.  To do so, it
needs a hook: a reason why the target or victim should be subject to
its invitations and temptations.  Deconstructed, Fear reveals the
strengths and weaknesses of the individual through the delivery of
information, misinformation, absurdity, and ridicule.

In the larger world of planetary political conflict, Fear is a card
typically played by Russia against the United States in the rivalry of
these two planetary political brothers.  It reflects socio-cultural
differences between these peoples, who need to recognize each other's
macro-social differences in order to build upon their common humanity.

That which invokes fear is very different for these two cultures.  The
relative indiscipline of Americans subjects us to fail to recognize
misinformation, subjects us to fail to differentiate information from
misinformation.  The relative discipline of Russians is to not
entertain information as relative to the self and the mind.  

In this differentiation, discipline and indiscipline are virtual
variables to be substituted with essays on economy and behavior and
situation and cultural origins.  It is not the differentiation of
discpline in the small as familiar to American individuals.  In one
avenue of development -- perhaps the primary avenue for development --
it is wealth versus poverty subject to culture and history.  

We could find threads of story within our own situation to illustrate
the difference leisure affords to individual presence of mind and
discipline in the discernment of information as objectively positive
or negative (true or untrue), or subjective and therefore objectively
indeterminate (input to an objective process).

\bye
