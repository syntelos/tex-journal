\input preamble

Perhaps the psychological flaw in the story amounts to no more than a
latent sense of exhaustion.  Unlike many other children I knew, I grew
up with the subliminal sense of voice-power in the observation or
regard for the difference between my mother, Lois, and her mother,
Ruth.  My grandfather, George, was our common source of light and
pride.  It was probably with a sense of expression that he once pulled
out a handy pocket pen light to read a menu in a restaurant.  An
expression never again performed for my grandmother's periodic review
of it.  And while Ruth was the force that balanced her husband, it was
my mother whose strength impressed me as more personally relevant than
my grandmother's.  Relevant in the sense of voice that is not unlike a
teenage sense of identity.  This I have always included in the heart I
inherited, the one that I would describe as a bit exhausting.

While it's fun to hypothesize about psychological impacts and
influences, it's far more significant to me that the early childhood
studies of sexual differentiation recognized and appreciated the life
force of the female -- the differentiation of which proposes balancing
and supporting over a background of impression and faculty.  Hence the
two halves of one notion that might resonate more with males than
females.

It is relatively late in life that I arrive at the displacement of
force hypothesis.  The leisurely or bourgeois character of modern life
affects males more than females, whose bodies would afflict them with
a number of imperatives (in comparison).  Or at least this analysis
has something to offer: the psychological effect of life force
displacement, and its relation to far less significant (however
charming) psychological theories.  For example, my mother's craziness
impacted me in my sense of exhaustion (or tedium) rather than my sense
of my own force or facility (capacity) of my own body.  Despite
ranging freely as a child, my own sense of leisure degraded my sense
of force.  {\it (Note: economic leisure includes many related issues,
not laziness or carelessness or listlessness in the small but
inutility and incapacitance in the large).}  And other childhood
experiences out of doors likewise contributed to my
intellectuo-emotional sense of tedium.  Unfortunately these did not
propel me into study as would have been truely brilliant.

Point being that microscopes are typically mistaken for telescopes by
the slight and lazy hand reaching for a gratifying conclusion -- as
any leaf of branch or blade of grass.  Socially Lois is an astute
member of her communities and more than a bit of a bright light and a
force for good.  In her world, the slight and ready would rob her
blind as for any other one who sticks a head of the din of the ground.
The quietude of reason often fails to gain traction above the noise of
confusion.  Or so it has been in the world until very recently.

\bye
