\input preamble-pdf


The homo sapien needs to know.  Our children emerge into learning
machines within a handful of years.  In this moment of history, many
are lost to the doldrums of boredom, cutoff from the epistemological
nutrients necessary to an intellectual development adapted to a
technological world.  This principal effect of the state of our
metaphysical epistemology is in the debt of greed.  Out of a too small
jealousy held too closely, we protect our metaphysical worlds to our
ultimate deficit.  Many adolescents as well as adults are revolted by
reflections of ourselves, including intellect and education.

This social inhibition to the development of the proper study and
illumination of an epistemology of metaphysics is responsible for
subliminal, or ``psychosexual'', interpersonal combat that plays out
daily.  The psychosexual rationale is complex, and chiefly refers to
the region of resistence which the metaphysical field inhabits.  At
this kind of development, sexuality has fused into an association with
languages of body, suggestion, and intercourse such that causal
discoveries of intent are often lost to imaginary projections of
search as in the case of reproductive signaling.

This, of course, is a principal theme within the production and
interpretation of research and literature.  Unfortunately, the price
we have paid, and continue to pay, for our sublimations is too high.
It is a losing strategy, the compromise of discipline to indiscipline.

Generally, it fails the test of quality of occasion.  That is, the
principle of fairness of occasion embodies a device that may be
described as an outcome or solution to competing alternatives, and the
set of alternatives to an occasion of merit require comparability of
attributes, and in turn these objects of comparison require
comparability of quality.  This device applies equally well to all
metaphysical objects in and of world as may be called the discovery of
invariants to a system characterization, or simply scientific
modeling.

\bye
