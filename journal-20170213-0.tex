\input preamble

Who am I to write of Psychology and Sociology?  A reader, and one with
no more than passing familiarity with Freud and Jung and Neumann.  I
studied Economics and Computer Science among the vast libraries of
Columbia University.  Not the disciplined adept, the the reader who
approaches the discipline of comprehension.  

In these notes and essays one won't find a review of literature in any
particular field of professional research, but an accumulation of
observation described with the skill of a technologist and, ideally, a
scientist with an appreciation for the r\^ole science plays in the
study of soft systems.

Hard science is easy.  Physics studies systems having reproducible
properties and behavior.  Epistemologically, when a Physicist knows
something to be true, he or she knows a fact that can be proven
repeatedly and independently.  Soft science is hard.  A social
scientist, an anthropologist, must maintain the scientific principles
at a celestial altitude while studying studying systems that do not
repeat themselves.  This is a great challenge, the very conception of
which regularly proves too abstract for capture or comprehension.  For
example, maintaining the scientific discipline compels me to record
these facts and issues of my relationship to our subject, here, and
further to attempt to describe the discipline of ``soft science'' for
completeness.

In the history of thought known variously as epistemology, doctrine,
philosophy, metaphysics and logical positivism, we have developed arts
of communication that lead the reader to the comprehension of our
writing in the form of the sharing of thought.  This is important that
an individual may grasp within the moment of reading (study) the
concept developed throughout the tenure of research.  With these tools
we maintain and develop (and defend!) no less than human
consciousness.  The altitudes of consciousness achieved by humanity
are preserved by the art of communication, and this is as important
and precious as a newborn.  Technically, as significant as a
generation and the generations in its future, but the concept
overwhelms so the reachable and recognizable principle of the one is
upheld with the full force of human faith.


\bye
