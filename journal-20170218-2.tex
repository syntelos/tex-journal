\input preamble

The inner consciousness is remote to the outer consciousness of the
observer.  The following descent requires extensive framing and
description, with metaphorical and analogical traversals, to establish
the subject.  Carl Jung demonstrated the derivation of speech from the
intimacy of the inner consciousness in his autobiography.  Sigmund
Freud mapped the domain in a model that served his disciplines of
observation and interaction.  The purpose here is to develop elements
of thought that may identify and characterize the inner consciousness
for the benefit of its knowledge and experience.  

That the inner consciousness emerges within the womb has a cause of
knowledge with the dominance of the subjective nature of the inner
consciousness over the objective nature of the outer consciousness
known to childhood, and the existence of the trauma of the passage of
canal birth within the conscious human experience.  In the service of
this cause I would identify {\it eros}, $\varepsilon$, with the
relationship of the individual to its mother, $\mu$, and to love,
$\nu$. $$ \varepsilon = \mu + \nu. $$ This expression of $\varepsilon$
fails to complete or close the character of {\it eros}.  Any
considered study of the subject consistently describes or discovers
such complexity.  The primal forces of mating and reproduction and
rearing ($\sigma + \rho$) exist within the inner consciousness --
perhaps {\it genetic memory}. $$ \varepsilon = \nu + \sigma + \rho. $$

The nondeterministic equation has multiple forms, the relationship
among which is particular to the context.  In this case, the inner
consciousness, it is a form of juxtaposition that is
abstract. $$\displaylines{ \varepsilon = \mu + \nu
+ \rho, \cr \varepsilon = \nu + \sigma + \rho.}$$ The intent is that
the representation, above, is unique and invariant and leading to
understanding and thereby to comprehension.

The inner consciousness is first described as subjective, $\xi$,
emotional, $\theta$, and spiritual, $\zeta$. $$ \iota = \xi + \theta
+ \zeta$$

This approach to the recognition of the inner consciousness employs a
hyper - nondeterministic form to describe an existential duality which
also includes
$$ \iota = \mu + \nu + \upsilon $$ where $\sigma$ and $\mu$ and $\nu$
have been abstracted into the components of force including caring,
affection and {\it eros}.
$$\displaylines{ \iota = \xi + \theta + \zeta, \cr \iota = \mu + \nu
+ \upsilon.}$$

\bye
