\input preamble

Hiding in the camoflage of popular fiction is the place of fantasy in
life.  The need is to differentiate metaphor and fantasy in their
respective r\^oles.  In the case of each physical and metaphysical
facility, with exercise and study we learn to master each power of
facility and faculty.  Fantasy is a kind of metaphor.  It has a
specific socio-cultural coloring as ``probably unhealthy''.  Escape as
running from reality may indicate a problem.  Metaphor and fiction as
running to reality is doing work on problems.

When fantasy exchanges r\^oles with metaphor in life, a moral hazard
is entered into.  The moral semantic is found in the social injury or
psychological trauma that is due to or emerges from misplaced fantasy.
By analogy, the psychic injury due to humor is obvious to children and
therefore well known.  In contrast, psychic traumas are subtle, and
often individually sublimated or socially repressed.  As a trauma is a
kind of injury, the distinction being made is in subtlety and
recognition.  The subtle case may be known in time, where the effect
has become separated from its cause.  The practitioner of self
awareness masters the identification and characterization of such
issues with relative ease.

\bye
