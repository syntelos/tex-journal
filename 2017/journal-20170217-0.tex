\input preamble

{\global\let\reviews = \rightleftharpoons}
{\global\let\accepts = \leftarrow}
{\global\let\rejects = \rightarrow}
{\global\let\rellang         = \lambda}
{\global\let\relouter        = \Re}
{\global\let\relinner        = \aleph}
{\global\let\relterm         = \tau}
{\global\let\termclass       = \kappa}
{\global\let\termpoint       = \pi}
{\global\let\termouter       = \alpha}
{\global\let\terminner       = \beta}

The epistemology of metaphysics requires language, $\rellang$, and
elements of thought, $\relterm$.  A pair of hemispheres circumscribes
the elements of thought in spaces, $\termclass$, and points,
$\termpoint$.

$$
 \relterm = \{ \termclass + \termpoint \}
$$

Languages employ or operate over categorical thought space and thought
spatial points.

$$
 \rellang = \bigcup_{\termclass}^{\termpoint} \relterm
$$

The categorical thought, $\termclass$, possesses boundaries.  The term
``person'' represents a semantic class or space, and even an
individual person has representation that fails to collapse thought or
description to a point.

That element of thought which evades description to a critical degree
is represented by the concept of a point in thought, $\termpoint$.  By
critical degree it is intended that the communication of a thought
spatial point would require extensive framing of context and
construction of description as remains elusive to some readers.

Likewise over $\termclass$ a scale of comparison or measurement is
available in the extent of framing and description required of
communication.  It's useful to identify a particular language as a
particular and contiguous subspace.

$$
 \rellang = \bigcap_{\termclass}^{\termpoint} \relterm 
$$

This conception of language complements our experience of art.  The
coherence of expressive means ($\rellang$) is proportional to the
capacity for the sharing or communication of thought.

$$
 \bigcap_{\relinner}^{\relouter} \rellang
$$

Mathematics is populated by thought spatial points, as in the
following relation over the contextual frame operator, $\phi$.

$$
 \phi \accepts \bigcap_{\relinner}^{\relouter} \rellang
$$

The intersection of language, as defined, over comprehension and
understanding, as defined, is represented by the contextual frame
operator.  

The contextual frame operator relates to language in a profound
existence within the spectrum of comprehension and understanding.
This spectrum is expected to be recognized from the experience of
learning.  

The reference frame abstraction is familiar to Physics as
in

$$
 \phi\alpha
$$

representing the reference frame, $\phi$, of some logical object
represented by the term $\alpha$.  

In a metaphysical logic, the definition of abstraction as analogous to
a physical reference frame is faced with the challenge inherent in the
distinction of ``metaphysical logical object'' from ``metaphysical
term of logic''.  When the metaphysical object may or may not be
discreet, there is a cause of preference for ``logical object'' over
``term of logic''.  

The ``logical object'' depends on the understanding that is not
exclusive of recognition as ``object of'' (communication or thought).
Conversely, the ``term of logic'' does not.  Likewise, the abstraction
is only more elusive and is shared by framing and description and
analogy.

A definition of $\phi$ 

$$\displaylines{ \forall\ \phi \in \phi\alpha \cr \exists\ \pi \in \phi \mid \pi \vdash \alpha}
$$

may be performed otherwise with logical proximity, ``$\vdash$'', and
membership constraint, ``$\mid$'', read ``such that''.

\bye
