\input preamble

A mathematical logic of human existence with respect to the mind and
spirit is subject to the epistemological universe of consciousness.
In the outer consciousness of intellect we may perceive elements of
thought, while within the inner consciousness of force and emotion and
spirit we may perceive no more than awarenesses and abstractions.

This individual is first represented with the following
nondeterministic equation of being.
$$\displaylines{ \psi = \alpha \times \beta \times \gamma, \cr
 \psi = \alpha + \beta + \gamma.}$$

The {\it spiritual state of being} of the individual, $\psi$, exists
in a continuum between two forms.  The terms $\alpha$, $\beta$, and
$\gamma$ (may) represent the relationships of the individual to
itself, its family, and society respectively.  In the first
representation we have the positive effects of spiritual complements
to being ({\it e.g.} peace and love and affection).  In the second we
have the absence of spiritual complements to being.

This form of exposition is described as relativistic.  The terms of
this epistemological logic represent elements of thought concerning
the metaphysics of human existence, proposed as essential and
invariant within the framing and description as found.  The terms of
epistemological logic over this existential domain are employed to
various classifications of representation or levels of abstraction.

On another level of abstraction, the equation of being
$$\displaylines{ \psi = \alpha \times \beta \times \gamma, \cr \psi
= \alpha + \beta + \gamma,}$$ can be understood to represent inner
($\alpha$), outer ($\beta$), and social consciousness ($\gamma$).

The discipline of mathematical logic, therefore, is abstracted to the
maintainence of clarity of expository identity and composition.  The
epistemological discipline is found at the proposed invariance of
identities.  The metaphysical discipline is found within the humanity,
within the pleasure and joy and respect with which this effort is
performed.

\bye
