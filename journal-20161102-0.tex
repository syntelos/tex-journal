\documentclass{article}

\everymath{\displaystyle}

\begin{document}

\title{A concise review of the problem with our future}

\author{John D.H. Pritchard \thanks{jdp@syntelos.com}}

\date{\today}

\maketitle

\begin{abstract}
An economy and politics founded on economic devices without intellectual foresight is destined to failure.
\end{abstract}


\section{Introduction}

That the future of the United States is in the hands of our economics
is self evident.  The impact of our economics is physical for most
people, and metaphysical otherwise.  That is, its influence is felt
physically or psychicly.  That impact is defined by the character of
our economics.  While this character is substantially flawed, our
future lies in peril.

We are living in the future produced by previous generations.  When
Washington elected a strong economy in his act to create the Federal
Reserve Banking System, his generation left to future generations the
responsibility for those actions.  Today, in their distant future, we
are well situated to review what we have learned from the historical
record and available data.

Our economic theories betray our children, our future.  While our
economics possesses the tools to manage the economy for any purpose we
desire, our comprehension and acceptance of the power and facility of
those tools has been far too limited to realize a reasonable and
stable future.  

As I write this we have not yet entirely and squarely faced our
experience with the so-called Kochtopus realized in the Clinton -
Trump Presidential race of 2016.  It was the national - global
experience of economic self destruction which absolutely violated the
sense of fidelty to intellect and principle (available through wealth
and leisure) that has been the foundation of an economic myopia (false
sense of security).  This collective economic na\"{i}vet\'e nearly
yielded the destruction of the human historical substance and
significance of the United States -- the antithesis of human survival
and the principles on which the country was founded.

\section{Detail}


The human collective termed society, $S$, employs tools for its
founding, definition, and structure termed economy, $E$.

\[
 \{ G, E \}\ \subset\ S
\]

The human intellectual - political concern, $C$, for the human
situation that the actions of government ($G$) effect in both
immediate and future impact has had the following character.

\[
 E\ \leftarrow\ \lim_{C\ \rightarrow\ 0}
\]

That is, a fidelity to a subjective theory of humanity which begs
fidelity to humanity has reduced $C$ to zero.  This is a
representation of the observation of the eco-political government of
the United States at the moment, at the dawn of the twenty first
century.

As described above, the impact of eco-political government on the
future, $F$, is so profound as to warrant determination without
reservation.

\[
 E\ \rightarrow\ F.
\]

The substance and significance cannot be stated more concisely.

Examples abound.  Consider the global effect of intellectual capacity
dimished by poor education, and expand it to include the national self
destruction in the eco-political context we find ourselves in today.
Point examples include the loss to advancement of intellectual
diversity undiscovered within an environment that is seemingly focused
on its own destruction.  

The r\^ole of the individual in the pursuit of happiness as subject to
the limitations of intellect (described by Heidegger as an incapacity
for self comprehension) do not release him from an obligation to develop
and maintain a social awareness and thereby conscience.  His self
preservation in a future over his epistemological horizon depends on
it.

\section{Conclusion}

The United States is, itself, a moral hazard.  The North American
Peoples which preceeded it a guillotine in history.  Should we violate
this obligation, we are doubly damned.  We are damned to the immediate
actuality, as well as the shame of the historic failure.

To date our politics has reflected the vanity of this existence in our
juvenile forms of intellect as the worship or maintenance of the
principles of democracy.  These distractions from a politics of reason
have produced nothing less than a politics of self destruction.

Humanity as we know it exists on Earth uniquely.  If we destroy Earth,
we betray ourselves profoundly.  This is the human cause.  Self
preservation.  It requires intellectual application and effort on
behalf of each living person capable of voting within the political
process.  

The Principle of the Separation of Church and State may be held up to
distance our collective political power over economic actuality from
our individual need for self realization inherent in Religion and
Metaphysics.

\section{Notes}

We have too little experience with political actors disciplined in
government, in the application of intellect to foresight and oversight
with the exercise of power within society -- most substantially the
tools and expertise of reason that generations have collected and
distilled into the logic, mathematics, and science.

\section{Thanks}

The writer thanks the President of the United States, Barack Obama,
for his compassion.

\end{document}
