\input preamble

It's as simple as nature versus nurture, hot and cold, heart and
intellect.  The internal consciousness within has as much breadth and
depth as the external consciousness of intellect and knowledge.  This
fact permits the homo sapien to abandon what it may know as head or
mind or intellect in the face of necessity, albeit with the
possibility of a singular (small) or periodic (large) case of
cognitive dissonance.  That is, a separation in the harmonic symbiosis
between heart and mind is disorienting and painful and is accompanied
by a severe need for relatively intimate compassion and support (in a
clinical context).

The external is learned, it is experiential, it is situational
awareness at its most primitive extent.  When one is strong enough to
admit the social complex {\it (animus)} as a variable, to admit
consciousness as a variable, one is strong enough to realize that
knowledge is epistemological.  It is objective with a finite,
mechanical context, or subjective with a non-finite, organic human
context {(\it i.e. originates with the inner consciousness in the
body)}.

Additionally, the formal subjective context established here admits
the possibility to define the term {\it human being} as one having at
least a subliminal appreciation of epistemology -- in this context.

Furthermore, an example.  I know that anger is a small thing, an
energetic product of the ego to defend the self as the whole of the
interior and the exterior, and that with this knowledge I am both a
healthier person {\it (i.e. homo sapien)} and a better human being.

\bye
