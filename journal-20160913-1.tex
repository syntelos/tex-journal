\documentclass{article}

\begin{document}

\title{The case for the right to earth}

\author{John D.H. Pritchard \thanks{jdp@syntelos.com}}

\date{\today}

\maketitle

\begin{abstract}
The right to earth is more abstract, but not incompatible with the
right to property.  
\end{abstract}


\section{Objective}

To present the case for the right to earth implied by native
reservation territory.

\section{Subjective}

How can we find peace without recognizing our obligations as stewards
of this land, this country, the region og the planet over which we
have fought and murdered?  

It is not because the people who came before us don't claim the rights
of property that we are free from the obligation of stewardship.  Did
we not take up that r\^ole?

Such is the want that would apply the misplaced principle to native
lands.  Are we afraid of the nomadic?  Are we still salting their
earth that they can't live on it?  Indeed, we should not preserve
ourselves by recognizing the right of life, liberty and the pursuit of
happiness to not artificially exclude ourselves.  Those of us with
another sense of spirit and soul would bring greater health if we
would not project our dishealth onto every possible case or situation.
This is the history of these United States.  That old superstitions
and delusions overcome our conscious principle in rhetoric and
silence.

I expect that even the least among us will reject the superficial when
consciously illuminated.  This we have done, and in the moment of
collective understanding we must recognize the collective capacity to
do right by ourselves in the most inclusive sense.

One man's property is another man's earth where our family claims the
rights of earth.  These delineations have been made in our history.
There is no cause for the defense of one concept over the other.
There is only the case of actuality wherein one steward should oppose
another, and in such a case it would be right and just that a court of
law should hear and try the case with every degree of impartiality
afforded disputes of property or finance.  Otherwise we are lost,
separated from our obligations, and promoting discord and dishealth.

In these presents there can be no doubt.  And from these presents we
are obligated to recognize the character of the right of earth.  It is
free of every impediment and social claim save the natural order the
earth was born into.  It is an expanse marked only by the presence of
those creatures who live on it.  This is the right of earth.  It is
subtle in comparison with the fenced and developed right of property.
It is sublime, as it may appear as nothing to some.  We have eyes to
see.  We are not blind to the value of open spaces free of
technological development.  We depend on them for our freedom, spirit
and health.  Likewise we depend on the sublime presence of those who
would develop and preserve the spiritual within us, for it is no less
than that freedom we hold so dear.  

The trajectory of the arc of history lies in justice.  We have
established justice as a third of our self government, and for this we
are great.  When we betray that greatness -- in this context of
history and prehistory -- we betray an obligation as profound as to
render us ignorant and uneducated.  Such is the greatness of the
opportunity of the United States as the single greatest opportunity
facing humanity: the success of this experiment in self government.  




\end{document}
