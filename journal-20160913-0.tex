\documentclass{article}

\begin{document}

\title{Type theory for programming languages}

\author{John D.H. Pritchard \thanks{jdp@syntelos.com}}

\date{\today}

\maketitle

\begin{abstract}
Need to read Whitehead and Russel and work it back through Hilbert's
{\it Mathematical Logic}.  This note reviews a conception of
categorical type theory to bring to the literature for review.
\end{abstract}


\section{Objective}

Fixing the foundation of a Type system to an abstraction that
qualifies and quantifies type design products.

\section{Subjective}

The application of number theory to type systems was established in
Whitehead and Russel.  Hilbert's sentential logic puts another stich
in that fabric.  I haven't found as much as expected that implements
type systems relative to these foundations.  

We know programming language types as operands and objects, and
abstract into the logical members of software systems.  We know bit
string primitives as the operands of machine instructions.
Formulating a continuum from one to the other has been implemented
many times for the benefit of practical trade anaylsis.
Implementational details aside, what is the foundation of type design
analysis and what would a type system notation look like.  Why don't
we take types seriously.  They cost enough.

\section{Implementation}

The organizational domain maps into the space and time dimensions in
functions representing type systems, programming languages, and
software systems.  That is, the organizational domain is composed of
categorical objects of attribute sets representing a type system or a
programming language or a software system.  An object in the
organizational domain has operators for identity and comparison, as
well as perhaps composition.  

\section{Conclusion}

A type system design (methodology) founded in number theory maximizes
language design possibility space extent, and tests or confronts type
system design as categorical.

\appendix

\section{Notes}

The exercise employs the attribute set to encode information into a
domain.  It describes types as well as spatial information interfaces.

\end{document}
