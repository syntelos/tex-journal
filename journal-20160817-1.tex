\input preamble

The implicit religion of objectivism is the habit of a juvenile,
ruthless objectivity lacking in a reasonable and proportional
subjective context.  

A decent understanding of the scientific method requires at least a
subliminal appreciation of measurement theory.  The statistically
consistent measurement or observation replaces the person as
instrument.  When the person is the instrument, an advanced topic by
any standard, an explicit appreciation of the fact is required, today.

Lacking these understandings, our immersion in the world left to us by
the scientific revolution in Physics from 1985 to 1911 -- sixteen
years that moved humanity into a new millenium, subjectively speaking
-- has us still searching for a handle on the moment.  We adopt
rationality and objectivity like leaves rather than stones.  That is,
these things of religious conflict require study and understanding as
the tools that they are.

It is a shame in every available sense that we should fail to consider
these things seriously, that we should allow noisy juveniles to shy us
away from their recognition.  Isn't this the situation?  We like our
formal religions.  They serve us as cultural touchstones to the
magnitude of the spiritual, and constant sources of faith in each
other and humanity.  Some people, the noisy ones, have too little
faith.  They mock science and cause harm to those who disagree with
their views on the theories and results offered by science.

It is this kind of influence in society, implicit and explicit, direct
and derived, overt and covert, good and evil, that would hinder our
healthy development.  A healthy look at the tools in our hands, and a
healthy open look at the subjectivity that is the homo sapien would be
as natural as breathing for the homo sapien.

\bye
