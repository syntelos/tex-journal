\input preamble-png

The vast amount of astronomical data requires an association between
the pixel coordinate system and the world or physical coordinate
system.  For example, an array of spectral flux measurements needs to
be associated with wavelengths through some sort of dispersion
solution, or an array of time-resolved flux measurements needs to be
associated with actual times in UTC.  In the simplest cases, these
associations can be expressed in linear terms: $$ world\ coordinate =
pixel\ coordinate \times scale + offset$$ where scale is the pixel
size (in world coordinate units) and offset is a zero-point
correction.  In more common cases, however, the relation between pixel
coordinates and world coordinates is nonlinear.

\bigskip

\leftline{\hfil {\bf World coordinate systems representations}\hfilneg}
\leftline{\hfil {\bf within the FITS Format}\hfilneg}

\leftline{\hfil Hanish, Wells, 1988\hfilneg}


\bye
