\input preamble

It's the hardest problem I know, the mating problem.  Took me forever
to find the one.  I'm fifty one years old.  I found her more than two
years ago, and I'm still sitting on it waiting.  Thoroughly
depressing.  You don't know a thing until you find the one that takes
the cake.  Stand em up next to any other and compare and you always
take em home, the one.  

Of course, getting to the heart of the matter is first.  Discarding
the attributes that will become irrelevant.  For me that's just about
the entirety of the material domain.  Makes sense, since it's a scene
that plays out in the inner consciousness of the heart.  The outer
consciousness of the intellect is no more than an observer, and a
hinting program.  Hint: don't try em all on -- you'll never survive
it.  Or so it is in my case, at least.

Next up, the giving problem.  Some see love, friendship, and go
hostile.  Some see my complexity of thought and language and loose
their sense of trust.  I suppose I've always been standing with sea
legs on this score.  In my experience, which itself has been rejected
as different and therefore invalid, many people tend to be wary or
destructive of intelligence in others.  

From where do our most common social-psychic illnesses come?
Obviously, these are no more than the reflection or mimicking of the
hyper-competitive state of our society and culture, today.  

The state of sex in society includes the implicit sex versus the
explicit sex.  Masculinity clouded over by no need to discuss it.
It's implicit.  Femininity clouded over by a single need to discuss
it.  It's explicit.  

We don't discuss sex or mating to a degree that is substantial or
significant to our needs.  Speaking on my own behalf as a child, I
need to have a handle on it before it arrives.  I need to be able to
frame my thoughts and experiences without suffering.  In my experience
of society I see pain and suffering and happiness in material terms.
The strongest social force, that originating in sex, is largely
unknown.  As an adult I cannot blame anyone who relates the material
and the sexual to a transactional interface.  As a machine I don't
understand why sex is not listed in the primary economy with food and
energy.  It is this representation of the most literal and superficial
interpretation of society and culture that is among the knives
employed to outcast or discard individuals as unhealthy, in avoidance
of the mental health issue of understanding and compassion and sharing
and education.  It is unhealthy, but not a knife.  It is a prognosis
of need in a healthy society.

\bye
